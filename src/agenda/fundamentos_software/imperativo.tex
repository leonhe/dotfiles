% Created 2017-04-26 Wed 14:39
% Intended LaTeX compiler: pdflatex
\documentclass[11pt]{article}
\usepackage[utf8]{inputenc}
\usepackage[T1]{fontenc}
\usepackage{graphicx}
\usepackage{grffile}
\usepackage{longtable}
\usepackage{wrapfig}
\usepackage{rotating}
\usepackage[normalem]{ulem}
\usepackage{amsmath}
\usepackage{textcomp}
\usepackage{amssymb}
\usepackage{capt-of}
\usepackage{hyperref}
\usepackage[brazil, english, portuguese]{babel}
\author{Francesco Antonello Ferraro}
\date{\today}
\title{Paradigma Imperativo}
\hypersetup{
 pdfauthor={Francesco Antonello Ferraro},
 pdftitle={Paradigma Imperativo},
 pdfkeywords={},
 pdfsubject={},
 pdfcreator={Emacs 25.1.2 (Org mode 9.0.5)},
 pdflang={Portuguese}}
\begin{document}

\maketitle
\selectlanguage{english}
\begin{abstract}
This article aims to show some specificity of the imperative paradigm, furthermore showing an analisys of how it compares
with the functional paradigm. For a better visualizations and understanding, tables will be use to show its differences.
\end{abstract}
\selectlanguage{portuguese}
\begin{abstract}
Esse artigo visa demostrar a especificidades do paradigma imperativo, além de exibir uma análise de como o mesmo se contrapõe com o paradigma funcional. Para que a visualização e entendimento sejam claros serão utilizadas tabelas contrastando essas diferenças.
\end{abstract}



\section{Contextualização}
\label{sec:org9d1b70b}

Uma linguagem de programação é a padronização de métodos que tem como objetivo comunicar instruções um computador mantendo a mesma estrutura sintática. Não obstante, um paradigma de programação é a visão que o indivíduo que digita o código tem sobre a estrutura e execução de um programa. O que nos leva a concluir que diferentes linguagem fazem escolhas diferentes quanto a forma e compilação dos mesmos, dependendo da visão que o seu autor vê como o mais ajustado para enfrentar os seus desafios e problemas que a mesma tenta resolver.
Mas o que torna uma linguagem imperativa? Basicamente um programa encrito de maneira imperativa possui duas característica básica:

\begin{itemize}
\item Um estado do programa;
\item Instruções que alteram o estado do programa;

Por intruções pode se inferir o sentido lexical da palavra de mandar, ou imperar que uma atividade seja executada de forma autoritária. Uma frase que é usualmente  usada na literatura para decrever esse paradigma é

\begin{quote}
Primeiro faça isso, depois faça aquilo.
\end{quote}
\end{itemize}
\section{Linguagens Imperativas}
\label{sec:org6590f4c}
\end{document}
